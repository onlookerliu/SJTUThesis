%# -*- coding: utf-8-unix -*-
%%==================================================
%% thesis.tex
%%==================================================

% 双面打印
\documentclass[master, openright, twoside]{sjtuthesis}
% \documentclass[bachelor, openany, oneside, submit]{sjtuthesis}
% \documentclass[master, review]{sjtuthesis}
% \documentclass[%
%   bachelor|master|doctor, % 必选项
%   fontset=fandol|windows|mac|ubuntu|adobe|founder, % 字体选项
%   oneside|twoside,        % 单面打印,双面打印(奇偶页交换页边距,默认)
%   openany|openright,      % 可以在奇数或者偶数页开新章|只在奇数页开新章(默认)
%   english,                % 启用英文模版
%   review,     % 盲审论文,隐去作者姓名、学号、导师姓名、致谢、发表论文和参与的项目
%   submit      % 定稿提交的论文,插入签名扫描版的原创性声明、授权声明 
% ]

% 逐个导入参考文献数据库
\addbibresource{bib/thesis.bib}
% \addbibresource{bib/chap2.bib}

%# -*- coding: utf-8-unix -*-
% !TEX program = xelatex
% !TEX root = ../thesis.tex
% !TEX encoding = UTF-8 Unicode
\title{应用在误差下降中的一类Levenberg-Marquardt方法}
\author{刘\quad{}凯}
\advisor{范金燕教授}
% \coadvisor{范金燕教授}
\defenddate{2018年12月17日}
\school{上海交通大学}
\institute{数学科学学院}
\studentnumber{116071910015}
\major{计算数学专业}
\keywords{非线性最小二乘, Levenberg-Marquardt方法, 神经网络, 权重更新}

\englishtitle{APPLIED LEVENBERG-MARQUARDT METHOD FOR ERROR REDUCTION}
\englishauthor{\textsc{Kai Liu}}
\englishadvisor{Prof. \textsc{Jinyan Fan}}
% \englishcoadvisor{Prof. \textsc{Jinyan Fan}}
\englishschool{Shanghai Jiao Tong University}
\englishinstitute{\textsc{School of Mathematical Sciences} \\
  \textsc{Shanghai Jiao Tong University} \\
  \textsc{Shanghai, P.R.China}}
\englishmajor{Computational Mathematics}
\englishdate{Dec. 17th, 2018}
\englishkeywords{nonlinear optimization, Levenberg-Marquardt method, Neural Network training}

  % NOTE: the enclosed commands must be executed in preamble

\begin{document}

% 无编号内容:中英文论文封面、授权页
\maketitle

\makeatletter
\ifsjtu@submit\relax
  \includepdf{pdf/original.pdf}
  \cleardoublepage
  \includepdf{pdf/authorization.pdf}
  \cleardoublepage
\else
\ifsjtu@review\relax
% exclude the original claim and authorization
\else
  \makeDeclareOriginal
  \makeDeclareAuthorization
\fi
\fi
\makeatother

\frontmatter % 使用罗马数字对前言编号

% 摘要
%# -*- coding: utf-8-unix -*-
% !TEX program = xelatex
% !TEX root = ../thesis.tex
% !TEX encoding = UTF-8 Unicode
%%==================================================
%% abstract.tex for SJTU Master Thesis
%%==================================================

\begin{abstract}

非线性系统最常见的损失函数即最小二乘,例如在神经网络的分类预测过程中,优化问题等价于在最小二乘损失下寻找最优的权重问题。这篇文章中,我们提出了一种改进的Levenberg-Marquardt方法用来训练这些系统,并成功应用在了人工(递归)神经网络中。相比已有的传统算法,我们证明了这种算法在满足局部误差的条件下,能够显著减少训练次数,并且在常见的数据集上提供了数值实验的对比。此外,我们还提出了一些调整超参数的策略。

\end{abstract}

\begin{englishabstract}

Error reduction is quite common in many nonlinear system, such as neural network training and solve nonlinear equations. Optimization task in particular scenario, is to find an optimal weight sets with delicate algorithms. In this paper, we proposed a modified Levenberg-Marquardt method for nonlinear system training and significantly reduce the least square error in our artificial neural networks. The theoretical basis for this method is given and the performance difference with respect to several other learning algorithms are shown on some well known data sets. The result indicates a much faster convergence rate under the local error bound condition. Also, some damping strategies are suggested to meet specific needs.

\end{englishabstract}



% 目录、插图目录、表格目录
\tableofcontents
\listoffigures
\addcontentsline{toc}{chapter}{\listfigurename}     % 将插图目录加入全文目录
\listoftables
\addcontentsline{toc}{chapter}{\listtablename}      % 将表格目录加入全文目录
\listofalgorithms
\addcontentsline{toc}{chapter}{\listalgorithmname}  % 将算法目录加入全文目录

\include{tex/symbol} % 主要符号、缩略词对照表

\mainmatter % 使用阿拉伯数字对正文编号

% 原正文内容:
% %# -*- coding: utf-8-unix -*-
% !TEX program = xelatex
% !TEX root = ../thesis.tex
% !TEX encoding = UTF-8 Unicode
%%==================================================
%% chapter01.tex for SJTU Master Thesis
%%==================================================

%\bibliographystyle{sjtu2}%[此处用于每章都生产参考文献]
\chapter{引言}
\label{chap:intro}

这是上海交通大学(非官方)学位论文 \LaTeX 模板,当前版本是 \version 。

最早的一版学位模板是一位热心的物理系同学制作的。
那份模板参考了自动化所学位论文模板,使用了CASthesis.cls文档类,中文字符处理则采用当时最为流行的 \CJKLaTeX 方案。
我根据交大研究生院对学位论文的要求
\footnote{\url{http://www.gs.sjtu.edu.cn/policy/fileShow.ahtml?id=130}}
,结合少量个人审美喜好,完成了一份基本可用的交大 \LaTeX 学位论文模板。
但是,搭建一个 \CJKLaTeX 环境并不简单,单单在Linux下配置环境和添加中文字体,就足够让新手打退堂鼓。
在William Wang的建议下,我开始着手把模板向 \XeTeX 引擎移植。
他完成了最初的移植,多亏了他出色的工作,后续的改善工作也得以顺利进行。

随着我对 \LaTeX 系统认知增加,我又断断续续做了一些完善模板的工作,在原有硕士学位论文模板的基础上完成了交大学士和博士学位论文模板。

现在,交大学位论文模板SJTUTHesis代码在github
\footnote{\url{https://github.com/sjtug/SJTUThesis}}
上维护。
你可以\href{https://github.com/sjtug/SJTUThesis/issues}{在github上开issue}
、或者在\href{https://bbs.sjtu.edu.cn/bbsdoc?board=TeX_LaTeX}{水源LaTeX版}发帖来反映遇到的问题。

\section{使用模板}

\subsection{准备工作}
\label{sec:requirements}

要使用这个模板撰写学位论文,需要在\emph{TeX系统}、\emph{TeX技能}上有所准备。

\begin{itemize}[noitemsep,topsep=0pt,parsep=0pt,partopsep=0pt]
	\item {\TeX}系统:所使用的{\TeX}系统要支持 \XeTeX 引擎,且带有ctex 2.x宏包,以2017年或更新版本的\emph{完整}TeXLive、MacTeX发行版为佳。
	\item TeX技能:尽管提供了对模板的必要说明,但这不是一份“ \LaTeX 入门文档”。在使用前请先通读其他入门文档。
	\item 针对Windows用户的额外需求:学位论文模本分别使用git和GNUMake进行版本控制和构建,建议从Cygwin\footnote{\url{http://cygwin.com}}安装这两个工具。
\end{itemize}

\subsection{模板选项}
\label{sec:thesisoption}

sjtuthesis提供了一些常用选项,在thesis.tex在导入sjtuthesis模板类时,可以组合使用。
这些选项包括:

\begin{itemize}[noitemsep,topsep=0pt,parsep=0pt,partopsep=0pt]
	\item 学位类型:bachelor(学位)、master(硕士)、doctor(博士),是必选项。
	\item 中文字体:fandol(Fandol 开源字体)、windows(Windows 系统下的中文字体)、mac(macOS 系统下的华文字体)、ubuntu(Ubuntu 系统下的文泉驿和文鼎字体)、adobe(Adobe 公司的中文字体)、founder(方正公司的中文字体),默认根据操作系统自动配置。
	\item 英文模版:使用english选项启用英文模版。
	\item 盲审选项:使用review选项后,论文作者、学号、导师姓名、致谢、发表论文和参与项目将被隐去。
\end{itemize}

\subsection{编译模板}
\label{sec:process}

模板默认使用GNUMake构建,GNUMake将调用latemk工具自动完成模板多轮编译:

\begin{lstlisting}[basicstyle=\small\ttfamily, caption={编译模板}, numbers=none]
make clean thesis.pdf
\end{lstlisting}

若需要生成包含“原创性声明扫描件”的学位论文文档,请将扫描件保存为statement.pdf,然后调用make生成submit.pdf。

\begin{lstlisting}[basicstyle=\small\ttfamily, caption={生成用于提交的学位论文}, numbers=none]
make clean submit.pdf
\end{lstlisting}

编译失败时,可以尝试手动逐次编译,定位故障。

\begin{lstlisting}[basicstyle=\small\ttfamily, caption={手动逐次编译}, numbers=none]
xelatex -no-pdf thesis
biber --debug thesis
xelatex thesis
xelatex thesis
\end{lstlisting}

\subsection{模板文件布局}
\label{sec:layout}

\begin{lstlisting}[basicstyle=\small\ttfamily,caption={模板文件布局},label=layout,float,numbers=none]
├── LICENSE
├── Makefile
├── README.md
├── bib
│   ├── chap1.bib
│   └── chap2.bib
├── bst
│   └── GBT7714-2005NLang.bst
├── figure
│   ├── chap2
│   │   ├── sjtulogo.eps
│   │   ├── sjtulogo.jpg
│   │   ├── sjtulogo.pdf
│   │   └── sjtulogo.png
│   └── sjtubanner.png
├── sjtuthesis.cfg
├── sjtuthesis.cls
├── statement.pdf
├── submit.pdf
├── tex
│   ├── abstract.tex
│   ├── ack.tex
│   ├── app_cjk.tex
│   ├── app_eq.tex
│   ├── app_log.tex
│   ├── chapter01.tex
│   ├── chapter02.tex
│   ├── chapter03.tex
│   ├── conclusion.tex
│   ├── id.tex
│   ├── patents.tex
│   ├── projects.tex
│   ├── pub.tex
│   └── symbol.tex
└── thesis.tex
\end{lstlisting}

本节介绍学位论文模板中木要文件和目录的功能。

\subsubsection{格式控制文件}
\label{sec:format}

格式控制文件控制着论文的表现形式,包括sjtuthesis.cfg和sjtuthesis.cls。
其中,“cls”控制论文主体格式,“cfg”为配置文件。

\subsubsection{主控文件thesis.tex}
\label{sec:thesistex}

主控文件thesis.tex的作用就是将你分散在多个文件中的内容“整合”成一篇完整的论文。
使用这个模板撰写学位论文时,你的学位论文内容和素材会被“拆散”到各个文件中:
譬如各章正文、各个附录、各章参考文献等等。
在thesis.tex中通过“include”命令将论文的各个部分包含进来,从而形成一篇结构完成的论文。
对模板定制时引入的宏包,建议放在导言区。

\subsubsection{各章源文件tex}
\label{sec:thesisbody}

这一部分是论文的主体,是以“章”为单位划分的,包括:

\begin{itemize}[noitemsep,topsep=0pt,parsep=0pt,partopsep=0pt]
	\item 中英文摘要(abstract.tex)。前言(frontmatter)的其他部分,中英文封面、原创性声明、授权信息在sjtuthesis.cls中定义,不单独分离为tex文件。
不单独弄成文件。
	\item 正文(mainmatter)——学位论文正文的各章内容,源文件是chapter\emph{xxx}.tex。
	\item 附录(app\emph{xx}.tex)、致谢(ack.tex)、攻读学位论文期间发表的学术论文目录(pub.tex)、个人简历(resume.tex)组成正文后的部分(backmatter)。
参考文献列表由bibtex插入,不作为一个单独的文件。
\end{itemize}

\subsubsection{图片文件夹figure}
\label{sec:fig}

figure文件夹放置了需要插入文档中的图片文件(支持PNG/JPG/PDF/EPS格式的图片),可以在按照章节划分子目录。
模板文件中使用\verb|\graphicspath|命令定义了图片存储的顶层目录,在插入图片时,顶层目录名“figure”可省略。

\subsubsection{参考文献数据库bib}
\label{sec:bib}

目前参考文件数据库目录只存放一个参考文件数据库thesis.bib。
关于参考文献引用,可参考第\ref{chap:example}章中的例子。


% \include{tex/formulation}
% \include{tex/example}
% \include{tex/faq}
% \include{tex/summary}

% 论文正文分为以下章
%# -*- coding: utf-8-unix -*-
% !TEX program = xelatex
% !TEX root = ../thesis.tex
% !TEX encoding = UTF-8 Unicode
%%==================================================
%% chapter01.tex for SJTU Master Thesis
%%==================================================

%\bibliographystyle{sjtu2}%[此处用于每章都生产参考文献]
\chapter{引言}
\label{chap:introduction}

近年来,机器学习和人工智能的许多领域的发展
神经网络\footnote{多层感知机(MLP)为神经网络的一种实现}通过前向传播和误差反向传播(EBP)算法\footnote{反向传播算法一般通过梯度法来实现}可以具有很好的表达能力。

           % 引言
\include{tex/formulation}            % 问题介绍
\include{tex/proposal}               % 一类改进的Levenberg-Marquardt算法介绍
\include{tex/prove}                  % 一类改进的Levenberg-Marquardt算法收敛性分析
\include{tex/numerical}              % 数值实验结果与实际应用问题
\include{tex/parameter}              % 参数更新策略
\include{tex/summary}                % 全文总结


\appendix % 使用英文字母对附录编号

% 原附录内容,本科学位论文可以用翻译的文献替代。
% 
% \include{tex/app_setup}
% \include{tex/app_eq}
% \include{tex/app_cjk}
% \include{tex/app_log}

% 附录内容
\include{tex/app_NN}               % 神经网络拓扑结构以及解释性
\include{tex/app_AD}               % 解析导数与自动微分


\backmatter % 文后无编号部分 

% 参考资料
\printbibliography[heading=bibintoc]

% 致谢、发表论文、申请专利、参与项目、简历
% 用于盲审的论文需隐去致谢、发表论文、申请专利、参与的项目
\makeatletter

% "研究生学位论文送盲审印刷格式的统一要求"
% http://www.gs.sjtu.edu.cn/inform/3/2015/20151120_123928_738.htm

% 盲审删去删去致谢页
\ifsjtu@review\relax\else
  \include{tex/ack}         % 致谢
\fi

\ifsjtu@bachelor
  % 学士学位论文要求在最后有一个英文大摘要,单独编页码
  \include{tex/end_english_abstract}
\else
  % 盲审论文中,发表学术论文及参与科研情况等仅以第几作者注明即可,不要出现作者或他人姓名
  \ifsjtu@review\relax
    \include{tex/pubreview}
    \include{tex/projectsreview}  
  \else
    \include{tex/pub}       % 发表论文
    \include{tex/projects}  % 参与的项目
    % \include{tex/patents}   % 申请专利
    \begin{resume}
  \begin{resumesection}{基本情况}
    刘凯,1995 年 01 月生于 江西樟树。
  \end{resumesection}

  \begin{resumelist}{教育背景}
    \item 2016 年 09 月至今,上海交通大学,硕士研究生,计算数学专业
    \item 2012 年 09 月至 2016 年 06 月,上海交通大学,本科,应用数学专业
  \end{resumelist}

  \begin{resumesection}{研究兴趣}
    算法与优化
    \item Machine Learning and Deep Learning
    \item \LaTeX{} 排版
  \end{resumesection}

  \begin{resumelist}{联系方式}
    \item 地址: 上海市闵行区东川路 800 号,200240
    \item E-mail: \email{liu1995@sjtu.edu.cn}
  \end{resumelist}
\end{resume}
    % 个人简历
  \fi
\fi

\makeatother

\end{document}
